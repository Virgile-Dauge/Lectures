%%%%%%%%%%%%%%%%%%%%%%%%%%%%%%%%%%%%%%%%%
% Journal Article
% LaTeX Template
% Version 1.4 (15/5/16)
%
% This template has been downloaded from:
% http://www.LaTeXTemplates.com
%
% Original author:
% Frits Wenneker (http://www.howtotex.com) with extensive modifications by
% Vel (vel@LaTeXTemplates.com)
%
% License:
% CC BY-NC-SA 3.0 (http://creativecommons.org/licenses/by-nc-sa/3.0/)
%
%%%%%%%%%%%%%%%%%%%%%%%%%%%%%%%%%%%%%%%%%
%----------------------------------------------------------------------------------------
%	PACKAGES AND OTHER DOCUMENT CONFIGURATIONS
%----------------------------------------------------------------------------------------

\documentclass[twoside,twocolumn]{article}

\usepackage{blindtext} % Package to generate dummy text throughout this template

\usepackage[sc]{mathpazo} % Use the Palatino font
\usepackage[utf8]{inputenc}
\usepackage[T1]{fontenc} % Use 8-bit encoding that has 256 glyphs
\linespread{1.05} % Line spacing - Palatino needs more space between lines
\usepackage{microtype} % Slightly tweak font spacing for aesthetics

\usepackage[french]{babel} % Language hyphenation and typographical rules

\usepackage[hmarginratio=1:1,top=32mm,columnsep=20pt]{geometry} % Document margins
\usepackage[hang, small,labelfont=bf,up,textfont=it,up]{caption} % Custom captions under/above floats in tables or figures
\usepackage{booktabs} % Horizontal rules in tables

\usepackage{lettrine} % The lettrine is the first enlarged letter at the beginning of the text

\usepackage{enumitem} % Customized lists
\setlist[itemize]{noitemsep} % Make itemize lists more compact

\usepackage{abstract} % Allows abstract customization
\renewcommand{\abstractnamefont}{\normalfont\bfseries} % Set the "Abstract" text to bold
\renewcommand{\abstracttextfont}{\normalfont\small\itshape} % Set the abstract itself to small italic text

\usepackage{titlesec} % Allows customization of titles
\renewcommand\thesection{\Roman{section}} % Roman numerals for the sections
\renewcommand\thesubsection{\roman{subsection}} % roman numerals for subsections
\titleformat{\section}[block]{\large\scshape\centering}{\thesection.}{1em}{} % Change the look of the section titles
\titleformat{\subsection}[block]{\large}{\thesubsection.}{1em}{} % Change the look of the section titles

\usepackage{fancyhdr} % Headers and footers
\pagestyle{fancy} % All pages have headers and footers
\fancyhead{} % Blank out the default header
\fancyfoot{} % Blank out the default footer
\fancyhead[C]{Rapport d'avancement de Thèse $\bullet$ \today} % Custom header text
\fancyfoot[RO,LE]{\thepage} % Custom footer text

\usepackage{titling} % Customizing the title section

\usepackage{hyperref} % For hyperlinks in the PDF

\usepackage{graphicx}

\usepackage{csquotes}
\usepackage[style=ieee,backend=bibtex]{biblatex}

\bibliography{biblio}

%----------------------------------------------------------------------------------------
%	TITLE SECTION
%----------------------------------------------------------------------------------------

\setlength{\droptitle}{-4\baselineskip} % Move the title up

\pretitle{\begin{center}\normalsize Rapport d'avancement \\
\Huge\bfseries} % Article title formatting
\posttitle{\end{center}} % Article title closing formatting
\title{Systèmes Cyber-Physiques autonomes et communicants en milieux hostiles. Application à l’exploration par robots mobiles. } % Article title
\author{%
\textsc{Virgile Daugé} \\[1ex] % Your name
\normalsize L.O.R.I.A. \\ % Your institution
\normalsize \href{mailto:virgile.dauge@inria.fr}{virgile.dauge@inria.fr} % Your email address
%\and % Uncomment if 2 authors are required, duplicate these 4 lines if more
%\textsc{Jane Smith}\thanks{Corresponding author} \\[1ex] % Second author's name
%\normalsize University of Utah \\ % Second author's institution
%\normalsize \href{mailto:jane@smith.com}{jane@smith.com} % Second author's email address
}
\date{\today} % Leave empty to omit a date
\renewcommand{\maketitlehookd}{%

%----------------------------------------------------------------------------------------
%	Abstract
%----------------------------------------------------------------------------------------

\begin{abstract}
\noindent L'objectif de ce rapport est de présenter brièvement l'avancement de mes travaux.
Premièrement, un rapide état de l'art des technologies permettant l'exploration autonome sera dréssé.
Suivie par une analyse des solutions existantes ainsi que les axes de développement dégagés.
Ce document ne représente qu'un bref résumé et n'est donc pas exhaustif.
\end{abstract}
}

%----------------------------------------------------------------------------------------

\begin{document}

% Print the title
\maketitle

%----------------------------------------------------------------------------------------
%	ARTICLE CONTENTS
%----------------------------------------------------------------------------------------

\section{Introduction}

\lettrine[nindent=0em,lines=3]{L}a robotique autonome est en plein essor, rendu possible à la fois
par l'arrivée de capteurs compacts et perfomants (cameras stéréos, Lidars...) et l'augmentation de la puissance de calcul embarquée.
Avant même d'effectuer une mission spécifique, un robot doit être capable de percevoir et d'intéragir avec son environnement.
Dans une grande majorité des systèmes actuels, cette perception de l'environement est déléguée à un ou plusieurs éléments extérieurs
(GPS, Systèmes de captures de mouvements, connaissances à priori, intervention humaine). Ceci n'est pas envisagable dans de nombreuses situations,
pour des raison d'inaccessibilité, de coûts, de dangerosité ou encore d'absence de connaissances préalables. Afin de rendre un système réellement autonome,
il est nécessaire de mettre en place des mécanismes de perception basés uniquement sur les capacités internes du système, les différents capteurs intégrables
ainsi que la puissance de calcul disponible. En effets, certaines tâches nécéssitent d'êtres effectuées en permanance à une fréquence suffisament élévée.
C'est le cas par exemple de la détection et l'évitement d'obstacles,
où la fréquence de ralentissement est directement liée à la vitesse de déplacement du système.
La première nécessité est de donner au système cyber-physique la capacité de cartographier son environnement
(même à un niveau de détails faible) et de se positionner au sein de cette carte
\footnote{Principalement appelé Simultaneous Localisation and Mapping \cite{durrant-whyte_simultaneous_2006}\cite{bailey_simultaneous_2006}}
%------------------------------------------------

\section{Etat de L'art}
Le SLAM étant une tâche complexe, il est souvent composé de nombreux algorithmes interagissant ensemble.
Une solution complète repose souvent sur le travail de plusieurs équipes, et est souvent adaptée à un cadre bien particulier.
L'enchevêtrement de solutions actuellement connues forme un ensemble très hétérogène et difficile à analyser.
Il m'a cependant parut pertinent de les diviser en deux grandes catégories : Plausibilité maximum \footnote{Maximum likelihood}
et Intégration au sein d'un graphe \footnote{Graph SLAM aussi appelé Smoothing and mapping}.
\subsection{Maximum Likelihood}
L'approche de plausibilité maximale consiste à analyser des données de capteurs à chaque étape, afin de determiner
la solution la plus probable et de l'intégrer directement à la carte, souvent par fusion directe, interdisant tout
repositionnement ou calcul postérieur. C'est le cas notamment de la solution la plus performante \cite{zhang_visual-lidar_2015}
eu égart des critères du benchmark de référence (KITTI) \cite{_kitti_????}. Solution basée sur une estimation de déplacement
combinant l'analyse des frames d'une caméra par Visual Odometry \cite{nister_visual_2004} ainsi qu'une comparaison de nuages de
points de Lidars par l'agorithme d'Iterative Closest Point \cite{pomerleau_review_2015}. L'avantage de ce type de méthode consiste en
sa "simplicité", et son coût réduit en termes calculs. L'inconvénient majeur réside dans l'impossibilité de corriger une erreur passée.
En effet, une fois intégrée à la carte, une mesure est considérée comme valide et servira de point de référence pour les estimations futures,
pouvant mener à une accumulation d'erreurs plus importante.

\subsection{Graph SLAM}
Afin de permettre une intégration rafinable des estimations successives de mouvements, une approche probabiliste a été dévellopée.
L'idée principale est d'appliquer une loi de probabilité à plusieurs variables (loi jointe) à travers les variables pour chaque étape
temporelle (extended Kalman filters) \cite{arulampalam_tutorial_2002}. Permettant ainsi de minimiser l'impact du bruit présent dans les mesures,
tout en rafinant l'estimation de carte et de trajectoire au fur et à mesure de la progression. L'inconvénient principal de cette méthode réside dans
la nécessité de mettre à jour l'intégralité du réseau à chaque étape, menant à un coût en calculs ne permettant pas son application en temps réel avec
les capacitées limitées embarquées dans un robot, surtout si l'on considère l'accumulation des mesures au fur et à mesure de l'exploration.
Afin de palier à ce problème, une équipe du Georgia Institute of Technology a dévellopé une méthode itérative\cite{kaess_isam:_2008} basée sur des réseau Bayesiens
et utilisant un graphe de factorisation factorisant la distribution de probabilités. En exploitant de surcroit les possibilités d'optimisation offertes
par les matrices creuses, cela permet de mettre à jour plus efficacement et rapidement le graphe. De plus, il est seulement nécessaire de réaliser
des calculs pour les nouvelles mesures et non plus l'intégralité des observations depuis le début de l'expérience.S'il est possible d'utiliser ces algorithmes en temps réel,
il reste toutefois difficile d'obtenir une estimation correcte de la covariance nécessaire à l'ajustement des mesures entres elles.
Dans le cas de l'ICP, certaines solutions analytiques permettent d'obtenir une estimation en 2D \cite{censi_accurate_2007} et en 3D \cite{prakhya_closed-form_2015}.
Cependant, les covariances estimées sont parfois erronées, tant en direction qu'en échelle. Ce qui conduira nécessairement à une mauvaise intégration des mesures dans le graphe.


%------------------------------------------------

\section{Results}
\blindtext

%------------------------------------------------

\section{Discussion}
\subsection{Current state}
\blindtext
\subsection{Possibles enhancements}
\blindtext

\printbibliography
\end{document}

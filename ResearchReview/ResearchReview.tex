\documentclass[twoside,twocolumn]{article}

\usepackage[utf8]{inputenc}
\usepackage[T1]{fontenc}

\usepackage[english]{babel}
\usepackage{geometry}

\usepackage{abstract} % Allows abstract customization
\renewcommand{\abstractnamefont}{\normalfont\bfseries} % Set the "Abstract" text to bold
\renewcommand{\abstracttextfont}{\normalfont\small\itshape} % Set the abstract itself to small italic text


\usepackage{hyperref} % For hyperlinks in the PDF

\usepackage{csquotes}
\usepackage[style=ieee,backend=bibtex]{biblatex}

\bibliography{Thesis}

\title{Research review on SLAM}

\author{%
\textsc{Virgile Daugé}  \\[1ex] % Your name
\normalsize University of Lorraine \\ % Your institution
\normalsize \href{mailto:virgile.dauge@inria.fr}{virgile.dauge@inria.fr} % Your email address
}

\begin{abstract}
\noindent  Goal : present what I have learned so far in SLAM.
\end{abstract}

\begin{document}
\maketitle

\section{Introduction}
Qu'est-ce que le slam ?
principes de base + fondations historiques

\section{Difficultés}
Domaine vaste et très développé :
termes confus (SLAM, SAM, VisualSlam,LidarSlam...)
Beaucoup d'étapes d'algorithmes utilisés dans les différentes approches.
Difficulté de juger de la validité des propositions (pas de benchmarks/comparaisons avec d'autres approches dans la litérature.)

\section{Décomposition en sous-problèmes}
\subsection{Localisation ET mapping}
Deux problèmes distincs sont addressés ici, la localisation et la cartographie.
La localisation (à minima l'estimation du mouvement) est la première étape,
et est essentielle dans la construction de la carte.
Il s'agit là d'estimer le mouvement réalisé par le capteur (et/ou)
l'environnement de la manière la plus précise possible, afin d'être par la
suite capable de construire la carte par superposition des données de capteurs
(caméra, lidar, wifi, sonar, capteur environnemental etc...).
Toutefois comme vous pouvez l'imaginer, ni la localisation ni la cartographie
ne sont des problèmes triviaux, il est donc préférable de les séparer en tâches
plus simples.
\subsection{Décomposition plus détaillée}
\subsubsection{mesure}
\subsubsection{association des données}
\subsubsection{lissage/optimisation de la trajectoire}
\subsubsection{correction de la dérive}
à court terme en recalant les mesures régulièrement sur la géometrie de
l'environnement scanné par exemple.
Au long terme en essayant de reconnaitre les lieux déjà visités.
\subsubsection{création de la carte}
Cela consiste notamment en la superposition des données issues des capteurs
(images, nuages de points, mesures de qualité de l'air etc ...).
Il y a de nombreuses façons de procéder, sans même compter les différentes
optimisations nécessaire à l'exploitabilité de la carte dans des conditions
réelles. En effet, une carte composée d'un nuage de point précis peut vite
demander plus de ressources mémoires/stockage que ce que la plupart des
systèmes actuels sont à même de fournir (à moins de disposer d'un supercalculateur.)

\subsubsection{amélioration de la carte}

Il y a ici de nombreuses optimisations possibles, tant en terme de consomation de ressources,
d'amélioration offline de la précision, d'ajout de contenu par extrapolation

Opérations couteuses

décimation
génération de mesh
classification de points
detection d'objets













\end{document}

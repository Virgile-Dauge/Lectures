%%%%%%%%%%%%%%%%%%%%%%%%%%%%%%%%%%%%%%%%%
% Journal Article
% LaTeX Template
% Version 1.4 (15/5/16)
%
% This template has been downloaded from:
% http://www.LaTeXTemplates.com
%
% Original author:
% Frits Wenneker (http://www.howtotex.com) with extensive modifications by
% Vel (vel@LaTeXTemplates.com)
%
% License:
% CC BY-NC-SA 3.0 (http://creativecommons.org/licenses/by-nc-sa/3.0/)
%
%%%%%%%%%%%%%%%%%%%%%%%%%%%%%%%%%%%%%%%%%
%----------------------------------------------------------------------------------------
%	PACKAGES AND OTHER DOCUMENT CONFIGURATIONS
%----------------------------------------------------------------------------------------

\documentclass[twoside,twocolumn]{article}

\usepackage{blindtext} % Package to generate dummy text throughout this template

\usepackage[sc]{mathpazo} % Use the Palatino font
\usepackage[utf8]{inputenc}
\usepackage[T1]{fontenc} % Use 8-bit encoding that has 256 glyphs
\linespread{1.05} % Line spacing - Palatino needs more space between lines
\usepackage{microtype} % Slightly tweak font spacing for aesthetics

\usepackage[english]{babel} % Language hyphenation and typographical rules

\usepackage[hmarginratio=1:1,top=32mm,columnsep=20pt]{geometry} % Document margins
\usepackage[hang, small,labelfont=bf,up,textfont=it,up]{caption} % Custom captions under/above floats in tables or figures
\usepackage{booktabs} % Horizontal rules in tables

\usepackage{lettrine} % The lettrine is the first enlarged letter at the beginning of the text

\usepackage{enumitem} % Customized lists
\setlist[itemize]{noitemsep} % Make itemize lists more compact

\usepackage{abstract} % Allows abstract customization
\renewcommand{\abstractnamefont}{\normalfont\bfseries} % Set the "Abstract" text to bold
\renewcommand{\abstracttextfont}{\normalfont\small\itshape} % Set the abstract itself to small italic text

\usepackage{titlesec} % Allows customization of titles
\renewcommand\thesection{\Roman{section}} % Roman numerals for the sections
\renewcommand\thesubsection{\roman{subsection}} % roman numerals for subsections
\titleformat{\section}[block]{\large\scshape\centering}{\thesection.}{1em}{} % Change the look of the section titles
\titleformat{\subsection}[block]{\large}{\thesubsection.}{1em}{} % Change the look of the section titles

\usepackage{fancyhdr} % Headers and footers
\pagestyle{fancy} % All pages have headers and footers
\fancyhead{} % Blank out the default header
\fancyfoot{} % Blank out the default footer
\fancyhead[C]{Running title $\bullet$ May 2016} % Custom header text
\fancyfoot[RO,LE]{\thepage} % Custom footer text

\usepackage{titling} % Customizing the title section

\usepackage{hyperref} % For hyperlinks in the PDF

\usepackage{graphicx}

\usepackage{csquotes}
\usepackage[style=ieee]{biblatex}

\bibliography{biblio}

%----------------------------------------------------------------------------------------
%	TITLE SECTION
%----------------------------------------------------------------------------------------

\setlength{\droptitle}{-4\baselineskip} % Move the title up

\pretitle{\begin{center}\normalsize Comments on : \\
\Huge\bfseries} % Article title formatting
\posttitle{\end{center}} % Article title closing formatting
\title{Handbook of robotics \\ - chapter 36 - \\ World Modeling} % Article title
\author{%
\textsc{Virgile Daugé}\thanks{A thank you or further information} \\[1ex] % Your name
\normalsize University of Lorraine \\ % Your institution
\normalsize \href{mailto:virgile.dauge@inria.fr}{virgile.dauge@inria.fr} % Your email address
%\and % Uncomment if 2 authors are required, duplicate these 4 lines if more
%\textsc{Jane Smith}\thanks{Corresponding author} \\[1ex] % Second author's name
%\normalsize University of Utah \\ % Second author's institution
%\normalsize \href{mailto:jane@smith.com}{jane@smith.com} % Second author's email address
}
\date{\today} % Leave empty to omit a date
\renewcommand{\maketitlehookd}{%

%----------------------------------------------------------------------------------------
%	Abstract
%----------------------------------------------------------------------------------------

\begin{abstract}
\noindent The way to represent the world is very influent in scanning and modeling results.
 This book chapter is listing different existing solutions.
\end{abstract}
}

%----------------------------------------------------------------------------------------

\begin{document}

% Print the title
\maketitle

%----------------------------------------------------------------------------------------
%	ARTICLE CONTENTS
%----------------------------------------------------------------------------------------

\section{Models for Indoors and structured environements}
\subsection{Occupancy grids}
The space is divaded into cells, each cell have is own probability of being an obstacle.
Mainly 2D grid of obstacle probability. Can be used in 3D. Used a lot for Slam.
High memory requirements (directly depending on the resolution of the grid).
Strongly sensor dependant (difficult to use with sonar for example cause is measurment area is quite large).

\subsection{Line maps}
Extracting lines from the environement, mainly by mean squares estimation.
Require less memory.
Scale better with the size of the environement.
More acurate (not discretisation's resolution dependant).
Not a closed-form solution.

\subsection{Topological Maps}
The environement is represented by a graph-like structure, in witch nodes are locally distinguishables places.
A way to use it is to compute a voronoi graph, showing paths free from obstacles.
Not really suited for mapping but can be usefull for navigation.

\subsection{Landmark-Based Maps}
Lighter than grid map, used in FASTSLAM

%------------------------------------------------

\section{Natural environements}
\subsection{Elevation Grids}
Using 2D grid of cells, where each cell contains an elevation.
Good for representing unconsistent terrain.
Bad for multi-level environemments.
The cell size (and grid resolution) can be adaptative according to the distance to the robot for exemple).
It's easy to use with Lidar (each point returned by the lidar can be a cell delimiter)

\subsection{3D grids and points sets/point clouds}
3D dense occupancy grids cost a lot of memory.

Point cloud consist of storing each points returned by a sensor according to the same reference pose.
Point clouds are more extensibles and scalable. However, the amount of data can be quite huge in prolongated scans.

\subsection{Meshes}
Representation surfaces with a set of polygones.
Compact representation, and there is a lot of available mesh simplification algorythms.
However, the extraction of surfaces can be really difficult in complex environements.
Difficult to represent vegetation with continuous surfaces.
Good for human visualisation.

\subsection{Cost maps}
Representing the space by the estimated cost of passing through.
Common way for rovers is to compute the angle of the slope.
Not really relevant for drones.

\subsection{Semantic Attributes}
Learning to recognize landmarks.
Can be viewed as classifictation problem.
Simple discrimination between obstacles and free region for navigation.
Difficult to pick good resolution for classification, as it is strongly dependant to the type of the environement.

%------------------------------------------------

\section{Dynamic Environements}
Dynamic object filtering by using feature-based tracking algorithm.

%------------------------------------------------

\section{Discussion}

\printbibliography
\end{document}

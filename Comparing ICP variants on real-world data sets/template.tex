%%%%%%%%%%%%%%%%%%%%%%%%%%%%%%%%%%%%%%%%%
% Journal Article
% LaTeX Template
% Version 1.4 (15/5/16)
%
% This template has been downloaded from:
% http://www.LaTeXTemplates.com
%
% Original author:
% Frits Wenneker (http://www.howtotex.com) with extensive modifications by
% Vel (vel@LaTeXTemplates.com)
%
% License:
% CC BY-NC-SA 3.0 (http://creativecommons.org/licenses/by-nc-sa/3.0/)
%
%%%%%%%%%%%%%%%%%%%%%%%%%%%%%%%%%%%%%%%%%
%----------------------------------------------------------------------------------------
%	PACKAGES AND OTHER DOCUMENT CONFIGURATIONS
%----------------------------------------------------------------------------------------

\documentclass[twoside,twocolumn]{article}

\usepackage{blindtext} % Package to generate dummy text throughout this template

\usepackage[sc]{mathpazo} % Use the Palatino font
\usepackage[utf8]{inputenc}
\usepackage[T1]{fontenc} % Use 8-bit encoding that has 256 glyphs
\linespread{1.05} % Line spacing - Palatino needs more space between lines
\usepackage{microtype} % Slightly tweak font spacing for aesthetics

\usepackage[english]{babel} % Language hyphenation and typographical rules

\usepackage[hmarginratio=1:1,top=32mm,columnsep=20pt]{geometry} % Document margins
\usepackage[hang, small,labelfont=bf,up,textfont=it,up]{caption} % Custom captions under/above floats in tables or figures
\usepackage{booktabs} % Horizontal rules in tables

\usepackage{lettrine} % The lettrine is the first enlarged letter at the beginning of the text

\usepackage{enumitem} % Customized lists
\setlist[itemize]{noitemsep} % Make itemize lists more compact

\usepackage{abstract} % Allows abstract customization
\renewcommand{\abstractnamefont}{\normalfont\bfseries} % Set the "Abstract" text to bold
\renewcommand{\abstracttextfont}{\normalfont\small\itshape} % Set the abstract itself to small italic text

\usepackage{titlesec} % Allows customization of titles
\renewcommand\thesection{\Roman{section}} % Roman numerals for the sections
\renewcommand\thesubsection{\roman{subsection}} % roman numerals for subsections
\titleformat{\section}[block]{\large\scshape\centering}{\thesection.}{1em}{} % Change the look of the section titles
\titleformat{\subsection}[block]{\large}{\thesubsection.}{1em}{} % Change the look of the section titles

\usepackage{fancyhdr} % Headers and footers
\pagestyle{fancy} % All pages have headers and footers
\fancyhead{} % Blank out the default header
\fancyfoot{} % Blank out the default footer
\fancyhead[C]{Running title $\bullet$ May 2016} % Custom header text
\fancyfoot[RO,LE]{\thepage} % Custom footer text

\usepackage{titling} % Customizing the title section

\usepackage{hyperref} % For hyperlinks in the PDF

\usepackage{graphicx}

\usepackage{csquotes}
\usepackage[style=ieee]{biblatex}

\bibliography{biblio}

%----------------------------------------------------------------------------------------
%	TITLE SECTION
%----------------------------------------------------------------------------------------

\setlength{\droptitle}{-4\baselineskip} % Move the title up

\pretitle{\begin{center}\normalsize Comments on : \\
\Huge\bfseries} % Article title formatting
\posttitle{\end{center}} % Article title closing formatting
\title{\citetitle{pomerleau_comparing_2013}} % Article title
\author{%
\textsc{Virgile Daugé}\thanks{A thank you or further information} \\[1ex] % Your name
\normalsize University of Lorraine \\ % Your institution
\normalsize \href{mailto:virgile.dauge@inria.fr}{virgile.dauge@inria.fr} % Your email address
%\and % Uncomment if 2 authors are required, duplicate these 4 lines if more
%\textsc{Jane Smith}\thanks{Corresponding author} \\[1ex] % Second author's name
%\normalsize University of Utah \\ % Second author's institution
%\normalsize \href{mailto:jane@smith.com}{jane@smith.com} % Second author's email address
}
\date{\today} % Leave empty to omit a date
\renewcommand{\maketitlehookd}{%

%----------------------------------------------------------------------------------------
%	Abstract
%----------------------------------------------------------------------------------------

\begin{abstract}
\noindent Description of ICP test protocols and experiments on several open Data sets. Discuss about ICP problems and solutions. Optimisation of ICP parameters.
\end{abstract}
}

%----------------------------------------------------------------------------------------

\begin{document}

% Print the title
\maketitle

%----------------------------------------------------------------------------------------
%	ARTICLE CONTENTS
%----------------------------------------------------------------------------------------

\section{Introduction}

\lettrine[nindent=0em,lines=3]{T}They highlight the need for a protocol for comparing ICP variants, due to the huge number of "coocking" variants and the actual difficulty to implement/test/compare them without re-doing the whole job.
They present the Iterativee closest point algorythm, who consist in matching two sets of points to deduce the transformation between them and the 6 DOF motion of the sensor.

%------------------------------------------------

\section{Methods}
They created a modular ICP chain, decoupling differents tasks:
\begin{itemize}
  \item data filtering,
  \item matching,
  \item Outlier Filtering
  \item Error minimisation
\end{itemize}

This chain is available in their library hosted on Github :\href{https://github.com/ethz-asl/libpointmatcher}{ethz-asl/libpointmatcher}.
They also provide standardized interfaces between each step.
They use Stop and Scan technique.
%------------------------------------------------

\section{Results}
\subsection{Overlap sensitivity}
Trought experiements, they highlight the registration sensitivity to scan overlap. More the overlap between scan is important, better will be the result.
ICP is working well when several scans are made in the same room, but is encounting some difficulties when scan is taken from a opening area like a door.
\subsection{ICP parameters}
There is a lot of parameters all along the ICP chain whatever the types of treatments choosen, often depending on the sensor.
For example, if we choose to keep only close points, we need to determine wich distance is the relevant one for discrimination.
Another example, if we choose to randomly sub-sampling pointClouds, we need to determine how many points are sufficent for good registration, allowing to reduce computation requirements.
It's possible to optimize those paramters in a fixed project.
\subsection{Point-to-point vs point-to-plane}
Point to plane variant seems to be more efficient in interior/structured buildings. However, in natural/unstructured environement, the point-to-plane is not that relevant.

\subsection{Number of points}
Depending on the sensor/algorythm, there is a number of points where additionnal points will only cost more computation time for a very low gain in precision.
%------------------------------------------------

\section{Discussion}
\subsection{Current state}
Their library seems quite complete, i really need to do some testing on it.
\subsection{Possibles enhancements}
Find a way to choose relevant points instead of a random selection can be a real imporvement.
Maybe working on new types of landmarks, like more complex/recognizable can allow better precision.

Using more complex primitives (like basic geometric shapes) is an interessant track.
Learning to recognize relevant points/structures in a pointcloud can maybe improve the whole process.

\printbibliography
\end{document}

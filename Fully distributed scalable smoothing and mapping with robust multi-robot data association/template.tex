%%%%%%%%%%%%%%%%%%%%%%%%%%%%%%%%%%%%%%%%%
% Journal Article
% LaTeX Template
% Version 1.4 (15/5/16)
%
% This template has been downloaded from:
% http://www.LaTeXTemplates.com
%
% Original author:
% Frits Wenneker (http://www.howtotex.com) with extensive modifications by
% Vel (vel@LaTeXTemplates.com)
%
% License:
% CC BY-NC-SA 3.0 (http://creativecommons.org/licenses/by-nc-sa/3.0/)
%
%%%%%%%%%%%%%%%%%%%%%%%%%%%%%%%%%%%%%%%%%
%----------------------------------------------------------------------------------------
%	PACKAGES AND OTHER DOCUMENT CONFIGURATIONS
%----------------------------------------------------------------------------------------

\documentclass[twoside,twocolumn]{article}

\usepackage{blindtext} % Package to generate dummy text throughout this template

\usepackage[sc]{mathpazo} % Use the Palatino font
\usepackage[utf8]{inputenc}
\usepackage[T1]{fontenc} % Use 8-bit encoding that has 256 glyphs
\linespread{1.05} % Line spacing - Palatino needs more space between lines
\usepackage{microtype} % Slightly tweak font spacing for aesthetics

\usepackage[english]{babel} % Language hyphenation and typographical rules

\usepackage[hmarginratio=1:1,top=32mm,columnsep=20pt]{geometry} % Document margins
\usepackage[hang, small,labelfont=bf,up,textfont=it,up]{caption} % Custom captions under/above floats in tables or figures
\usepackage{booktabs} % Horizontal rules in tables

\usepackage{lettrine} % The lettrine is the first enlarged letter at the beginning of the text

\usepackage{enumitem} % Customized lists
\setlist[itemize]{noitemsep} % Make itemize lists more compact

\usepackage{abstract} % Allows abstract customization
\renewcommand{\abstractnamefont}{\normalfont\bfseries} % Set the "Abstract" text to bold
\renewcommand{\abstracttextfont}{\normalfont\small\itshape} % Set the abstract itself to small italic text

\usepackage{titlesec} % Allows customization of titles
\renewcommand\thesection{\Roman{section}} % Roman numerals for the sections
\renewcommand\thesubsection{\roman{subsection}} % roman numerals for subsections
\titleformat{\section}[block]{\large\scshape\centering}{\thesection.}{1em}{} % Change the look of the section titles
\titleformat{\subsection}[block]{\large}{\thesubsection.}{1em}{} % Change the look of the section titles

\usepackage{fancyhdr} % Headers and footers
\pagestyle{fancy} % All pages have headers and footers
\fancyhead{} % Blank out the default header
\fancyfoot{} % Blank out the default footer
\fancyhead[C]{Running title $\bullet$ May 2016} % Custom header text
\fancyfoot[RO,LE]{\thepage} % Custom footer text

\usepackage{titling} % Customizing the title section

\usepackage{hyperref} % For hyperlinks in the PDF

\usepackage{graphicx}

\usepackage{csquotes}
\usepackage[style=ieee,backend=bibtex]{biblatex}

\bibliography{Thesis}

%----------------------------------------------------------------------------------------
%	TITLE SECTION
%----------------------------------------------------------------------------------------

\setlength{\droptitle}{-4\baselineskip} % Move the title up

\pretitle{\begin{center}\normalsize Comments on : \\
\Huge\bfseries} % Article title formatting
\posttitle{\end{center}} % Article title closing formatting
\title{\citetitle{cunningham_fully_2012}} % Article title
\author{%
\textsc{Virgile Daugé}\thanks{A thank you or further information} \\[1ex] % Your name
\normalsize University of Lorraine \\ % Your institution
\normalsize \href{mailto:virgile.dauge@inria.fr}{virgile.dauge@inria.fr} % Your email address
%\and % Uncomment if 2 authors are required, duplicate these 4 lines if more
%\textsc{Jane Smith}\thanks{Corresponding author} \\[1ex] % Second author's name
%\normalsize University of Utah \\ % Second author's institution
%\normalsize \href{mailto:jane@smith.com}{jane@smith.com} % Second author's email address
}
\date{\today} % Leave empty to omit a date
\renewcommand{\maketitlehookd}{%

%----------------------------------------------------------------------------------------
%	Abstract
%----------------------------------------------------------------------------------------

\begin{abstract}
\noindent  This  paper’s  focus  is  on  the  front-end  data
association  system. DDF-SAM algorithm provides a decentralized communication and inference
scheme, but did not address the crucial issue of data association. RANSAC-based, approach for
performing the between-robot data associations and initialization of relative frames of reference. They focus  on  the  multi-robot  perception
problem,  and  present  an  experimentally  validated  end-to-end  multi-robot  mapping  framework
\end{abstract}
}

%----------------------------------------------------------------------------------------

\begin{document}

% Print the title
\maketitle

%----------------------------------------------------------------------------------------
%	ARTICLE CONTENTS
%----------------------------------------------------------------------------------------

\section{Introduction}

\lettrine[nindent=0em,lines=3]{T}The inspiration for DDF-SAM \cite{cunningham_ddf-sam:_2010} was decentralized data
fusion (DDF). Most of the work is comming from collaborative localization communnity \cite{bailey_decentralised_2011,bahr_consistent_2009}.

Other approaches :
\begin{itemize}
 \item Smoothing and mapping \cite{andersson_c-sam:_2008}
 \item relative pose graphs \cite{kim_multiple_2010}
 \item particle filters \cite{howard_multi-robot_2006,carlone_rao-blackwellized_2010}
 \item manifold representaions \cite{howard_multirobot_2006}
 \item divide and conquer with locally optimized maps \cite{ni_multi-level_2010}
\end{itemize}

Single Robot data association has been studied extensively from
Nearest Neighbor \cite{} to more complex Joint Compatibility Branch
and Bound \cite{neira_data_2001}.

The RANSAC \cite{fischler_random_1981} approach to data association
in the presence of outliers has become a staple algorithm in
computer vision for efficiently matching landmarks across frames, with variations to exploit
domain  structure,  such  as  GroupSAC  \cite{ni_groupsac:_2009}.

Recent  work
has addressed the decentralized resolution of inconsistencies
\cite{aragues_consistent_2011},  but  the  problem  had  yet  to  be  solved  sufficiently  for
real-time performance in robot systems

%------------------------------------------------

\section{Problem description}
We  focus  on  the  problem  in  which  a  robot
$r$
jointly
estimates its trajectory
$X^r$
and a map of landmarks
$L$
both
within its local sensor range, as well as landmarks observed
by neighboring robots.

We consider a graph where both robot poses
and landmarks are nodes and measurements are represented
by edges, and wish to solve for the most probable configura-
tion.

The  key
intuition   for   this   approach   to   decentralized   inference   is
solving  the  local  SLAM  problem  on  each  robot  and  then
distributing  condensed  versions  of  these  local  solutions  to
other robots, which can solve for the full neighborhood map.


%------------------------------------------------

\section{DDF-SAM}
Composed of 3 main modules :
\begin{description}
 \item[Local Mapping Module
 :]  optimizes  for  the  full  trajec-
 tory and landmark map, then compresses a local map
 for broadcast to neighboring robots.
 \item [Communications Module
 :]  updates  a  cache  of  con-
 densed maps from many robots.
 \item [Neighborhood Mapping Module
 :] jointly estimates over
 landmarks in the robot’s neighborhood graph and rel-
 ative coordinate frames to yield a
 neighborhood map
 .
\end{description}

%------------------------------------------------

\section{Multi-Robot Data asociation}
the multi-robot
data  association  module  uses  a  triangulation-based  robust
estimator  for  matching  feature  maps.

\subsection{triangle map matching}
A  direct  application  of  RANSAC  \cite{fischler_random_1981}  would  randomly
sample  (without  replacement)  two  points  from  the  neighborhood  landmarks
$L$
and  the  incoming  landmarks
$L^a$
as
putatives,  compute  the  corresponding  transformation  and
verify it by counting the number of matched landmarks.
Problem : there  are  exponentially  many  transformations  to
verify.

their approach  first  computes  a  Delaunay  triangulation
of  the  landmark  positions  in
$L$
and
$L^a$
,  and  then  applies
RANSAC on a set of putatives generated from the centroids
of  similar  triangles.

They choose to use a Delaunay triangulation as a geometric
feature  because  :
\begin{itemize}
 \item it  is  unique,
 \item invariant  to  reference  frame,
 \item and  biases  the  set  of  putatives  towards  conservative  data
       associations
 \item can be computed in $\Theta
       (
       n
       log
       n
       )$
\end{itemize}
Then they matches the two sets of Triangles with RANSAC \cite{fischler_random_1981}.

Possible enhancement : GroupSAC \cite{ni_groupsac:_2009} or PROSAC \cite{chum_matching_2005}.

\section{Experiments}
\subsection{softwares}

The  graphical  inference  and  optimization
engine  used  is  the  GTSAM  library,  with  local  optimization
performed  with  an  improved  version  of  the  iSAM \cite{kaess_isam:_2008}.

he  neighborhood  optimization  approach  uses
batch  Levenberg-Marquardt  optimization.

We  perform  2D
triangulation with the Triangle library \cite{shewchuk_triangle:_1996}.

PB : Not directly usable in 3D.

\subsection{environnement & hardware}
Landmarks are provided by a pole detector from laser range measurements.

\printbibliography
\end{document}

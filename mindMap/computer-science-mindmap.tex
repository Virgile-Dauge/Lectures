% Author: Till Tantau
% Source: The PGF/TikZ manual
\documentclass{article}

\usepackage{tikz}
\usetikzlibrary{mindmap,trees}
\usepackage{verbatim}

\begin{document}
\pagestyle{empty}

\begin{comment}
:Title: Computer science mindmap
:Tags: Manual, Mindmap

Version 1.09 of PGF/TikZ added a library for drawing mindmaps. Here's an example
from the manual.

| Author: Till Tantau
| Source: The PGF/TikZ manual

\end{comment}

\begin{tikzpicture}
 \path[mindmap,concept color=black,text=white]
 \begin{scope}[mindmap, concept color=orange, text=white]
  \node [concept] {Informatique}[clockwise from=-5]
  child {node [concept] (log) {M{\'e}thodes cat{\'e}goriques}}
  child {node [concept] (alg) {Algorithmique}}
  child {node [concept] (cod) {Compression \& transmission}}
  child {node [concept] (img) {Tra{\^i}tement des images}}
  child {node [concept] (opt) {Optimisation}}
  child {node [concept] (res) {R{\'e}seaux}};
 \end{scope}
 node[concept] {Robotics}
 [clockwise from=0]
 child[concept color=green!50!black] {
  node[concept] {environement}
  [clockwise from=90]
  child { node[concept] {algorithms} }
  child { node[concept] {data structures} }
  child { node[concept] {pro\-gramming languages} }
  child { node[concept] {software engineer\-ing} }
 }
 child[concept color=blue] {
  node[concept] {applied}
  [clockwise from=-30]
  child { node[concept] {databases} }
  child { node[concept] {WWW} }
 }
 child[concept color=red] { node[concept] {technical} }
 child[concept color=cyan] { node[concept] {coucou} }
 child[concept color=orange] { node[concept] {theoretical} };
\end{tikzpicture}\end{document}

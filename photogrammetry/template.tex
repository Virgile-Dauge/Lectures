%%%%%%%%%%%%%%%%%%%%%%%%%%%%%%%%%%%%%%%%%
% Journal Article
% LaTeX Template
% Version 1.4 (15/5/16)
%
% This template has been downloaded from:
% http://www.LaTeXTemplates.com
%
% Original author:
% Frits Wenneker (http://www.howtotex.com) with extensive modifications by
% Vel (vel@LaTeXTemplates.com)
%
% License:
% CC BY-NC-SA 3.0 (http://creativecommons.org/licenses/by-nc-sa/3.0/)
%
%%%%%%%%%%%%%%%%%%%%%%%%%%%%%%%%%%%%%%%%%
%----------------------------------------------------------------------------------------
%	PACKAGES AND OTHER DOCUMENT CONFIGURATIONS
%----------------------------------------------------------------------------------------

\documentclass[twoside,twocolumn]{article}

\usepackage{blindtext} % Package to generate dummy text throughout this template

\usepackage[sc]{mathpazo} % Use the Palatino font
\usepackage[utf8]{inputenc}
\usepackage[T1]{fontenc} % Use 8-bit encoding that has 256 glyphs
\linespread{1.05} % Line spacing - Palatino needs more space between lines
\usepackage{microtype} % Slightly tweak font spacing for aesthetics

\usepackage[english]{babel} % Language hyphenation and typographical rules

\usepackage[hmarginratio=1:1,top=32mm,columnsep=20pt]{geometry} % Document margins
\usepackage[hang, small,labelfont=bf,up,textfont=it,up]{caption} % Custom captions under/above floats in tables or figures
\usepackage{booktabs} % Horizontal rules in tables

\usepackage{lettrine} % The lettrine is the first enlarged letter at the beginning of the text

\usepackage{enumitem} % Customized lists
\setlist[itemize]{noitemsep} % Make itemize lists more compact

\usepackage{abstract} % Allows abstract customization
\renewcommand{\abstractnamefont}{\normalfont\bfseries} % Set the "Abstract" text to bold
\renewcommand{\abstracttextfont}{\normalfont\small\itshape} % Set the abstract itself to small italic text

\usepackage{titlesec} % Allows customization of titles
\renewcommand\thesection{\Roman{section}} % Roman numerals for the sections
\renewcommand\thesubsection{\roman{subsection}} % roman numerals for subsections
\titleformat{\section}[block]{\large\scshape\centering}{\thesection.}{1em}{} % Change the look of the section titles
\titleformat{\subsection}[block]{\large}{\thesubsection.}{1em}{} % Change the look of the section titles

\usepackage{fancyhdr} % Headers and footers
\pagestyle{fancy} % All pages have headers and footers
\fancyhead{} % Blank out the default header
\fancyfoot{} % Blank out the default footer
\fancyhead[C]{Running title $\bullet$ May 2016} % Custom header text
\fancyfoot[RO,LE]{\thepage} % Custom footer text

\usepackage{titling} % Customizing the title section

\usepackage{hyperref} % For hyperlinks in the PDF

\usepackage{graphicx}

\usepackage{csquotes}
\usepackage[style=ieee,backend=bibtex]{biblatex}

\bibliography{biblio}

%----------------------------------------------------------------------------------------
%	TITLE SECTION
%----------------------------------------------------------------------------------------

\setlength{\droptitle}{-4\baselineskip} % Move the title up

\pretitle{\begin{center}\normalsize Comments on : \\
\Huge\bfseries} % Article title formatting
\posttitle{\end{center}} % Article title closing formatting
\title{\citetitle{westoby_structure--motion_2012}} % Article title
\author{%
\textsc{Virgile Daugé}\thanks{A thank you or further information} \\[1ex] % Your name
\normalsize University of Lorraine \\ % Your institution
\normalsize \href{mailto:virgile.dauge@inria.fr}{virgile.dauge@inria.fr} % Your email address
%\and % Uncomment if 2 authors are required, duplicate these 4 lines if more
%\textsc{Jane Smith}\thanks{Corresponding author} \\[1ex] % Second author's name
%\normalsize University of Utah \\ % Second author's institution
%\normalsize \href{mailto:jane@smith.com}{jane@smith.com} % Second author's email address
}
\date{\today} % Leave empty to omit a date
\renewcommand{\maketitlehookd}{%

%----------------------------------------------------------------------------------------
%	Abstract
%----------------------------------------------------------------------------------------

\begin{abstract}
\noindent This paper outlines a low-cost, user-friendly photogrammetric technique for obtaining
high-resolution datasets at a range of scales, termed
‘
Structure-from-Motion
’
(SfM). The goal of my little review was to gain a basic understanding of known photogrammetry techniques, to be able indentify the impact of our positioning method in concrete applications.
I was mainly searching for an estimation of motion needed to cover and reconstruct 3D environement, in order to compare our work with well known odometry/SLAM solutions.
\end{abstract}
}

%----------------------------------------------------------------------------------------

\begin{document}

% Print the title
\maketitle

%----------------------------------------------------------------------------------------
%	ARTICLE CONTENTS
%----------------------------------------------------------------------------------------

\section{Introduction}

\lettrine[nindent=0em,lines=3]{I}n this paper, we report on an emerging, low-cost photogrammet-
ric method for high resolution topographic reconstruction, ideally
suited for low-budget research and application in remote areas.
‘
Structure-from-Motion
’
(SfM) operates under the same basic tenets
as stereoscopic photogrammetry, namely that 3-D structure can be
resolved from a series of overlapping, offset images.However,
it differs fundamentally from conventional photogrammetry, in that
the geometry of the scene, camera positions and orientation is solved
automatically without the need to specify a priori, a network of tar-
gets which have known 3-D positions. Instead, these are solved si-
multaneously using a highly redundant, iterative bundle adjustment
procedure, based on a database of features automatically extracted
from a set of multiple overlapping image. \\
the approach is most suited to sets of images with a
high degree of overlap

there exist few quantitative assessments of the quality of terrain
products derived from this approach.
\blindtext

%------------------------------------------------

\section{Methods}
Camera pose and scene geometry are reconstructed simultaneously through the automatic
identification of matching features in multiple images. Scale Invariant Feature Transform (SIFT) is used for features extraction.
\\
Following keypoint identification and descriptor assignment, the sparse bundle adjustment system Bundler (
Snavely et al., 2008)is used to estimate camera pose and extract a low-density or
‘
sparse
’
point cloud.
\\
Keypoints in multiple images are matched using approximate nearest neighbour (
Arya et al., 1998) and Random Sample Consensus (RANSAC;Fischler and Bolles, 1987) algorithms, and
‘
tracks
’
linking specific keypoints in a set of pictures are established.
\\
Tracks
comprising a minimum of two keypoints and three images are used
for point-cloud reconstruction, with those which fail to meet these
criteria being automatically discarded
\\
transient features such as people moving across the
area of interest are automatically removed from the dataset before
3-D reconstruction begins
\\
Keypoint correspondences place constraints on camera pose orientation, which is reconstructed using a similarity transformation,
while minimisation of errors is achieved using a non-linear least-
squares solution
\\
Finally, triangulation is used to estimate the 3-D point positions and incrementally reconstruct scene geometry,
fixed into a relative coordinate system
\\
the transformation of SfM image-space coordinates to an absolute coordinate system
can be achieved using a 3-D similarity transform based on a small number of known
ground-control points (GCPs)
\\
it is often easier to deploy physical targets with a high
contrast and clearly defined centroid in the field before acquiring images
%------------------------------------------------

\section{Results}
After manual editing, the sparse dataset comprised 1.7×10e5
points, while the dense reconstruction produced 11.3×10e6 points; a 64-fold increase. This is comparable to the
TLS survey density with 11.7×10e6 survey points over the same area.
\\
No real error comparison ?

%------------------------------------------------

\section{Discussion}
As the above examples demonstrate, the apparent logistical advan-
tages of SfM (limited hardware needs and portability) are, at least in
part, offset by the lengthy processing times compared to
‘
data-ready
’
methods such as TLS or GPS. Keypoint descriptor extraction, matching,
and sparse and dense reconstruction algorithms are computationally
demanding.

\printbibliography
\end{document}

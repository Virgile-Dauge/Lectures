%%%%%%%%%%%%%%%%%%%%%%%%%%%%%%%%%%%%%%%%%
% Journal Article
% LaTeX Template
% Version 1.4 (15/5/16)
%
% This template has been downloaded from:
% http://www.LaTeXTemplates.com
%
% Original author:
% Frits Wenneker (http://www.howtotex.com) with extensive modifications by
% Vel (vel@LaTeXTemplates.com)
%
% License:
% CC BY-NC-SA 3.0 (http://creativecommons.org/licenses/by-nc-sa/3.0/)
%
%%%%%%%%%%%%%%%%%%%%%%%%%%%%%%%%%%%%%%%%%
%----------------------------------------------------------------------------------------
%	PACKAGES AND OTHER DOCUMENT CONFIGURATIONS
%----------------------------------------------------------------------------------------

\documentclass[twoside,twocolumn]{article}

\usepackage{blindtext} % Package to generate dummy text throughout this template

\usepackage[sc]{mathpazo} % Use the Palatino font
\usepackage[utf8]{inputenc}
\usepackage[T1]{fontenc} % Use 8-bit encoding that has 256 glyphs
\linespread{1.05} % Line spacing - Palatino needs more space between lines
\usepackage{microtype} % Slightly tweak font spacing for aesthetics

\usepackage[english]{babel} % Language hyphenation and typographical rules

\usepackage[hmarginratio=1:1,top=32mm,columnsep=20pt]{geometry} % Document margins
\usepackage[hang, small,labelfont=bf,up,textfont=it,up]{caption} % Custom captions under/above floats in tables or figures
\usepackage{booktabs} % Horizontal rules in tables

\usepackage{lettrine} % The lettrine is the first enlarged letter at the beginning of the text

\usepackage{enumitem} % Customized lists
\setlist[itemize]{noitemsep} % Make itemize lists more compact

\usepackage{abstract} % Allows abstract customization
\renewcommand{\abstractnamefont}{\normalfont\bfseries} % Set the "Abstract" text to bold
\renewcommand{\abstracttextfont}{\normalfont\small\itshape} % Set the abstract itself to small italic text

\usepackage{titlesec} % Allows customization of titles
\renewcommand\thesection{\Roman{section}} % Roman numerals for the sections
\renewcommand\thesubsection{\roman{subsection}} % roman numerals for subsections
\titleformat{\section}[block]{\large\scshape\centering}{\thesection.}{1em}{} % Change the look of the section titles
\titleformat{\subsection}[block]{\large}{\thesubsection.}{1em}{} % Change the look of the section titles

\usepackage{fancyhdr} % Headers and footers
\pagestyle{fancy} % All pages have headers and footers
\fancyhead{} % Blank out the default header
\fancyfoot{} % Blank out the default footer
\fancyhead[C]{Running title $\bullet$ May 2016} % Custom header text
\fancyfoot[RO,LE]{\thepage} % Custom footer text

\usepackage{titling} % Customizing the title section

\usepackage{hyperref} % For hyperlinks in the PDF

\usepackage{graphicx}

\usepackage{csquotes}
\usepackage[style=ieee]{biblatex}

\bibliography{biblio}

%----------------------------------------------------------------------------------------
%	TITLE SECTION
%----------------------------------------------------------------------------------------

\setlength{\droptitle}{-4\baselineskip} % Move the title up

\pretitle{\begin{center}\normalsize Comments on : \\
\Huge\bfseries} % Article title formatting
\posttitle{\end{center}} % Article title closing formatting
\title{\citetitle{zhang_low-drift_2017}} % Article title
\author{%
\textsc{Virgile Daugé}\thanks{A thank you or further information} \\[1ex] % Your name
\normalsize University of Lorraine \\ % Your institution
\normalsize \href{mailto:virgile.dauge@inria.fr}{virgile.dauge@inria.fr} % Your email address
%\and % Uncomment if 2 authors are required, duplicate these 4 lines if more
%\textsc{Jane Smith}\thanks{Corresponding author} \\[1ex] % Second author's name
%\normalsize University of Utah \\ % Second author's institution
%\normalsize \href{mailto:jane@smith.com}{jane@smith.com} % Second author's email address
}
\date{\today} % Leave empty to omit a date
\renewcommand{\maketitlehookd}{%

%----------------------------------------------------------------------------------------
%	Abstract
%----------------------------------------------------------------------------------------

\begin{abstract}
\noindent This paper explain how LOAM algorithm is working, and present some results.
This method, coupled with IMU, try to achieve both low drift in motion estimation and low computational complexity.
\end{abstract}
}

%----------------------------------------------------------------------------------------

\begin{document}

% Print the title
\maketitle

%----------------------------------------------------------------------------------------
%	ARTICLE CONTENTS
%----------------------------------------------------------------------------------------

\section{Introduction}
\lettrine[nindent=0em,lines=3]{T}he problem addressed here is to achieve SLAM without accurate pose estimation (as GPS).
They don't consider Loop Closure, as they consider it unnecessary for mapping floor of a building.
They are dividing SLAM problem into 2 algorithms :
\begin{itemize}
  \item Odometry estimation at high frequency (but low fidelity)
  \item Registration of Point Cloud (lower frequency)
\end{itemize}

They are presenting their work in a high level of details.

They said that when the lidar scanning rate is high compared to this motion, motion distortion is neglectable and ICP \cite{pomerleau_comparing_2013} can be used.
Other methods exists for removing motion distortion \cite{hong_vicp:_2010} and \cite{moosmann_velodyne_2011} vor velodyne lidar.
Using IMU in multimodal system can compensate intermittent GPS \cite{scherer_river_2012}.

IMU and loop closure alows to build large indoors maps in underground mines : \cite{yoshida_efficient_2014}.

They used 4 different lidars for testing purposes.

%------------------------------------------------

\section{Methods}
\subsection{Lidar odometry}
\subsubsection{feature point extraction}
They select feature point on sharp edges and planar surfaces patches.
they evaluate smoothness of local surface, then sort them, considering smoother points as planar and sharpest points as edges.
\subsubsection{Feature point correspondence}
When adding more points to the pointcloud, they try to find ones in features region as lines and planar areas.
\subsubsection{Motion estimation}
They estimate a geometric relationship between a point and his feature line/planar area.
And Computing the sum of all geometric transformations give the estimated motion of the lidar.





%------------------------------------------------

\section{Results}
\blindtext

%------------------------------------------------

\section{Discussion}
\subsection{Current state}
\blindtext
\subsection{Possibles enhancements}
\blindtext

\printbibliography
\end{document}
